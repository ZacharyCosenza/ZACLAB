\documentclass[12pt]{article}

\usepackage[utf8]{inputenc}
\usepackage[T1]{fontenc}
\usepackage{geometry}
\usepackage{bm} %bold stuff
\geometry{a4paper}

\title{\vspace{-20mm} \textbf{Book Report 4: \emph{1848: A Year of Revolution} by Mike Rapport}}

\begin{document}
\date{April 1, 2021}
\maketitle
\thispagestyle{empty} %get rid of page numbering on this page only

This book report will be about the eponymous year of 1848, also known as the Springtime of the Peoples/Nations. Mike Rapport wrote this highly readable but detailed account of the famous revolutions that swept across Europe during 1848-1850. This was quite a fun book to read especially coming off the heals of \emph{The Shadow Emperor} which went into too much detail about a single man and instead focused on the collections of conservative nobles, liberal lawyers, and socialist agitators that made up the bulk of this revolution as well as their ideologies. This was a book as much about long-running economic, social, and political trends as about the men (usually men) that fought for and against these ideas.

\vspace{5mm}

The basic idea behind the revolutions of 1848 was that the growth of democratic, liberal, and socialist ideas, especially in the increasingly important middle class, came into contact with the old order of conservatives, the church, and the multi-nationality empires that spanned Europe. That last part is important because, as Mike makes clear in this book, nationalism played an important role in politicizing people against the government, and often helped to unite radicals and moderates. It also helped destroy the revolutions and their gains in many empires as national consolidation and ethnic rivalry was used by conservatives to hold onto power. Fun!

\vspace{5mm}

It was also a lot of fun to see the differences and similarities between the politics of 1848 and 2021. The terms used, like liberal, socialist, conservative, are familiar, but the programs they represented are often quite different. The conservatives were reactionary (of course) but also supported multi-ethnic empires and supranational organization of peoples, similar to how many liberals and cosmopolitans of today do. Thus, nationalists were often on the side of liberals and the general left, whereas now they often ally with the right. Workers and students were often the closest of allies in the cities of Europe, whereas now they are often (to their own detriment) pitted against each other by economic elites. Basically, this book details the opening shots of modern politics.

\end{document}