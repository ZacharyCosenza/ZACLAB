\documentclass[12pt]{article}

\usepackage[utf8]{inputenc}
\usepackage[T1]{fontenc}
\usepackage{geometry}
\usepackage{bm} %bold stuff
\geometry{a4paper}

\title{\vspace{-20mm} \textbf{Book Report 3: \emph{The Shadow Emperor: A Biography of Napoleon III}}}

\begin{document}
\date{March 1, 2021}
\maketitle
\thispagestyle{empty} %get rid of page numbering on this page only

Okay everyone we're doing another emperor book. This time it's a historical biography of the first and only emperor of the Second French Empire: Louis Napoleon III, detailed in the biography \emph{The Shadow Emperor: A Biography of Napoleon III}. I don't know why all three of the books so far have been about empires but that's how it shakes out. 

\vspace{5mm}

The biography goes through the life and rule of the little known emperor \emph{of the French} Louis Napoleon III from his upbringing in the extended family of the Bonaparte family, through his early political and military failures in Italy as a twenty-something (if you haven't participated in a failed unification of a nation-state are you even living?), into his failed coups, into his (half brothers) successful coup, to his failed military adventures as emperor. Yes, what is remarkable about Napoleon III is just how much he failed, how his political ambitions were floated by his aristocratic wealth, and how much he prospered from the work of previous generations. If I'm being harsh on the man it's because he was a profoundly mediocre leader at a time when Europe was growing industrially and at peace. The best he could have done was nothing. He was, however, a fairly liberal and modern leader compared to the ultra-conservative leaders on the continent. This was during the backdrop of post-1848 revolution and the collapse of the stately quadrangle. And while he did manage to humiliate France during the Franco-Prussian War that fueled generational anger at the Germans, French society did reach some pretty amazing peaks during his \emph{Belle Epoque} 

\vspace{5mm}

The book itself is fairly good. I have trouble remembering names and his early life is full of the names of passersby that you will not remember and aren't important. Two other takeaways I have are (i) this book should be about August de Morney and not Napoleon III and (ii) there were way too many parties during the Second French Empire. I'd prefer to hear more about the growth of French industry, urbanization, labor market conditions, and the growth of new socialist and liberal republican movements than the flings the emperor had at St. Cloud. Perhaps that's why I'd prefer a social/economic history rather than a biography, and why that'll probably be the next book!

\end{document}