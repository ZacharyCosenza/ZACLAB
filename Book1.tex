\documentclass[12pt]{article}

\usepackage[utf8]{inputenc}
\usepackage[T1]{fontenc}
\usepackage{geometry}
\usepackage{bm} %bold stuff
\geometry{a4paper}

\title{\vspace{-20mm} \textbf{Book Report 1: \emph{A Memory Called Empire}}}

\begin{document}
\date{January 1, 2021}
\maketitle
\thispagestyle{empty} %get rid of page numbering on this page only

I hope everyone had a great new year! I know it's been a tough 2020 what with COVID, the election, and everything else that has been happening. I have really been enjoying this process of writing a blog post for everyone (that no one will read of course) so I am going to keep the ball rolling and start a book blog! This will just be a once-a-month post about a book. I am still working on the formatting but I'll just bang out a few paragraphs about the book and call it a day. Great let's get started! The book I spent \textbf{way} too long reading was \emph{A Memory Called Empire (Teixcalaan Book 1)} by Arkady Martine. To quote directly from Wikipedia: It follows Mahit Dzmare, the ambassador from Lsel Station to the Teixcalaanli Empire, as she investigates the death of her predecessor and the instabilities that underpin that society. The book won the 2020 Hugo Award for Best Novel. It's part political thriller, part detective story taking place in a space opera environment. 

\vspace{5mm}

I don't want to give away spoilers and also it's very late here so I'll make two points about the book that will hopefully convince some readers to give it a try. First, the world building is done in a really smart way. Oftentimes sci-fi and fantasy books/media focus too much on developing a large volume of material to consume. And why not? After all we enjoy learning about the world of Star Wars and Game of Thrones. But brevity is an underrated virtue in literature. Here, Martine amplifies only the critical elements of description; flurries about the futuristic city architecture, the algorithm that controls the police and transit system, the resplendence of the costumes of the citizenry of this galactic empire, much to the improvement of the story. The rest of left to the reader. My second point is that this book focuses on the exciting topic of ... \emph{institutional memory}? But isn't that the sign of a great writer? We get to see how two very different societies pass information, technical, interpersonal, cultural, from generation to generation. This is something not often explored in storytelling; how the dissemination of culture through time and space (joke intended), in the case of this Teixcalaanli Empire often poems, reinforces power structures and attitudes. 

\vspace{5mm}

That's all I got. Again this is mostly to keep me honest and also I wanted to keep it under one page. I think next month we will be talking about another sci-fi book; \emph{Dune}!

\end{document}