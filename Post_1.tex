\documentclass[12pt]{article}

\usepackage[utf8]{inputenc}
\usepackage[T1]{fontenc}
\usepackage{geometry}
\geometry{a4paper}

\title{\textbf{Part 1: Introduction to \emph{ZACLAB}}}

\begin{document}
\maketitle

What is this? What am I doing here? Why am I doing whatever this is? This is \emph{ZACLAB}, an open blog being hosted on \emph{Github}. I have chosen \emph{Github} because it is free, open, and is already the supporting infrastructure behind my PhD and personal work. I am here to post about data science, optimization, statistics, modeling and related things. The reasoning is multiple: (1) I would like to write more and learn more about \LaTeX, \emph{Github} and related things, (2) I would like to work on more varied subjects in the above scientific and mathematical topics, and (3) I need something to push me to do so. That last point is important, without directed goals, even on subjects not fully related to my work, I find I am paralyzed and unable to do \textbf{any} work. So hopefully the possibility that anyone is reading this now or in the future will keep me honest!

\vspace{5mm}

The structure of \emph{ZACLAB} will be simple. Every week I will release a PDF on one subject or another. Along with the PDF will be code (usually Python but sometimes \emph{MATLAB}, hence the name of this blog). I will try to focus on producing pieces of code that will help us dive deeply into the underlying subject, and will try to integrate these more closely into the PDF for ease of reading. This may end up being blocks of \emph{.m} or \emph{.py} files that can be run easily with the correct packages and functions, or \emph{Juypter} files (once I figure out how they work better). The names of the posts will be $n + D$ for the blog number $n$ and date of release $D$, will will be similar to other files so everything should be organized chronologically.

\vspace{5mm}

Cool! Next week we will be talking about some of the bread and butter of a field of optimization which is highly practical for \emph{expensive} optimization: \emph{Expected Improvement/EI} for modeling with Gaussian Process Models!

\end{document}