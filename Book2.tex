\documentclass[12pt]{article}

\usepackage[utf8]{inputenc}
\usepackage[T1]{fontenc}
\usepackage{geometry}
\usepackage{bm} %bold stuff
\geometry{a4paper}

\title{\vspace{-20mm} \textbf{Book Report 2: \emph{Dune}}}

\begin{document}
\date{February 1, 2021}
\maketitle
\thispagestyle{empty} %get rid of page numbering on this page only

Okay everything this report is about \emph{Dune} by Frank Herbert. The book is about Paul Atreides, the son of a powerful and respected Duke who is given control of the economically import planet of Arrakis by the emperor of their feudal civilization. They're in a typical Holy Roman Empire situation, where the central authority is the emperor but a significant amount of wealth and power is in the hands of local houses. One such house, the Harkonnens, conspire with the emperor to destroy the Atreidians. After Paul and his mother are practically the only survivors of a Harkonnen / Imperial attack, the create a new life with the local people in the deserts of Arrakis and manage to defeat the Harkonnen / Imperial forces. 

\vspace{5mm}

Honestly, I wasn't a big fan of \emph{Dune}. I found the focus on religious / mystical powers in the science fiction setting to be off putting and the style of writing to be slightly difficult to read (as opposed to \emph{A Memory Called Empire} which was strangely easy to read). I find that \emph{Dune} does a similar thing to technology that the \emph{Warhammer 40K} universe does. Technology is described as vast and science is unexplained, and it is largely hidden from the reader behind vagaries and omissions. Of course this is basically the definition of "soft" science fiction, and does aid in the feudal aesthetic (buildings of local materials, everyone wears robes, nuclear reactors are treated as precious family heirlooms, swords are more common than projectile weapons), but I have trouble wrapping my mind around how a society like that can operate. To be even more honest, I did not finish the book (80\% done). 

\vspace{5mm}

I did, however, like discussions of the ecology of Arrakis and the way that the Fremen of the planet survive. When writing about the natural cause and effect of the environment and how someone might survive (by creating an economy around water for example), Herbert's knowledge of these complicated systems shines. I hope that my not liking the book as much as most people do does not dissuade anyone from reading it. Next month I will be reading \emph{The Shadow Emperor: A Biography of Napoleon III} (a lot of emperors for some reason).

\end{document}